{
\def\sym#1{\ifmmode^{#1}\else\(^{#1}\)\fi}
\begin{tabular*}{1.0\hsize}{@{\hskip\tabcolsep\extracolsep\fill}l*{6}{c}}
\hline\hline
                &\multicolumn{1}{c}{(1)}&\multicolumn{1}{c}{(2)}&\multicolumn{1}{c}{(3)}&\multicolumn{1}{c}{(4)}&\multicolumn{1}{c}{(5)}&\multicolumn{1}{c}{(6)}\\
                &\multicolumn{1}{c}{Push, Shake, Throw}&\multicolumn{1}{c}{Slap}&\multicolumn{1}{c}{Twist Arm, Pull Hair}&\multicolumn{1}{c}{Punch}&\multicolumn{1}{c}{Choke, Burn}&\multicolumn{1}{c}{Threaten to attack with weapon}\\
\hline
Ascending       & -0.01557         &  0.01003         & -0.00549         & -0.00407         & -0.00385         & -0.00055         \\
                &(0.01355)         &(0.01386)         &(0.01306)         &(0.01302)         &(0.01155)         &(0.00921)         \\
\hline
Observations    &     6133         &     6134         &     6135         &     6134         &     6135         &     6135         \\
\hline\hline
\multicolumn{7}{p{1.0\linewidth}}{\footnotesize Notes: The table reports results from an experiment where we randomised the order of the frequency answer options. Ascending is an indicator of the order options displayed to the respondent in ascending order for questions related to the experience of violence asked using ACASI. The dependent variables take on the value 0 if the respondent did not experience the type of violence in the last 6 months, 1.5 if she experienced it once or twice and 3 if she experienced it three or more times. Standard errors are in parentheses. \sym{*} \(p<0.10\), \sym{**} \(p<0.05\), \sym{***} \(p<0.01\).}\\
\end{tabular*}
}
